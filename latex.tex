\documentclass{article}
\usepackage{amsmath}
\usepackage{graphicx}
\usepackage{siunitx}
\usepackage{url}
\usepackage{float} % To use the [H] specifier
\usepackage{hyperref}


\title{Hot Air Balloon Project}
\author{Thomas Xiao}
\date{}

\begin{document}

\maketitle

\section*{Introduction}
The concept of hot air ballooning is based on fundamental principles of buoyancy and the ideal gas law, where heating air within a balloon decreases its density, allowing it to rise above the cooler, denser air outside. In this project, we aim to design a hot air balloon capable of lifting a group of five individuals. 

\section*{Analysis}

\subsection*{Total Weight of Passengers}
For this project, we assume that the total weight of all passengers is approximately \( W_{\text{passengers}} = 362.9 \, \text{kg} \).

\subsection*{Weight of the Balloon Cabin}
The weight of the cabin required for a hot air balloon capable of carrying five passengers around that weight of 362.9  kg is estimated to be approximately \( W_{\text{cabin}} = 45 \, \text{kg} \), according to \textit{Napa Valley Aloft}, a hot air balloon service company.





\subsection*{Expression for the Mass of Hot Air Inside the Balloon}
From the ideal gas law, we have:
\[
PV = nRT
\]
where $P$ is the pressure, $V$ is the volume, $n$ is the amount of gas in moles, $R = 8.314 \, \mathrm{J/(mol \cdot K)}$ is the ideal gas constant, and $T$ is the temperature in Kelvin.

The number of moles $n$ of hot air inside the balloon is given by:
\[
n = \frac{m_h}{M}
\]
where $m_h$ is the mass of the hot air and $M$ is the molar mass of dry air ($M = 0.02897 \, \mathrm{kg/mol}$).

Thus, rearranging for $m_h$, we find:
\[
m_h = \frac{PV}{RT} M = \frac{0.02897PV}{8.314T} = \frac{0.00348PV}{T}
\]

\subsection*{Expression for the Mass of Cold Air Inside the Balloon}
For the cold air inside the balloon, we use the atmospheric temperature $T_c$, which depends on altitude $h$:
\[
T_c(h) = 288 - 0.0064h
\]
and the pressure:
\[
P(h) = 101.83 - 0.0107h
\]

The mass of cold air $m_c$ is then:
\[
m_c = \frac{P(h)V}{R T_c(h)} M =  \frac{0.00348P(h)V}{T_c(h)}
\]
The mass of cold air, unlike hot air in a controlled system, is a function of altitude, and thus atmospheric temperature and pressure are used.
\subsection*{Total Upward Force}
The upward force is generated by the difference in weight between the hot air and the surrounding cold air. The buoyant force $F_{\text{up}}$ acting on the balloon is:
\[
F_{\text{up}} = (m_c - m_h) g
\]
where $g = 9.81 \, \mathrm{m/s^2}$.
\section*{Balloon Shape and Dimensions}

The balloon design consists of a hemisphere on top of a curved truncated cone. This shape provides a rounded top, with a gradual taper towards the bottom. According to hot air balloon manufacturing company \textit{Cameron Balloons}, a model A-105 balloon that can lift at least 5 people is 61 feet across and 64 feet tall. Therefore, let $R$ = 10 meters = 32.8 feet and $H$ = 20 meters = 65.6 feet are applied, where \( R \) is the radius of both the hemisphere and the top of the truncated cone, and \( H \) being the total height of the balloon, including both the hemisphere and the cone. We choose these values not only for ease of calculation, but to also ensure that the balloon can withstand the weight of the passengers. Since both chosen dimensions are slightly bigger than real-life dimensions, we can safely assume that the modeled balloon will be able to withstand the weight.

\subsection*{Hemisphere (Top Part)}
The hemisphere sits on top and has a radius \( R = 10 \) meters. The equation of the hemisphere in terms of \( x \), \( y \), and \( z \) is:
\[
x = R  \sin(\phi) \cos(\theta)
\]
\[
y = R  \sin(\phi) \sin(\theta)
\]
\[
z = R \cos(\phi)
\]
where \( 0 \leq \theta \leq 2\pi \) and  \( 0 \leq \phi \leq \frac{\pi}{2} \)

\begin{figure}[H]
    \centering
    \includegraphics[scale=0.25]{hemi.png}
    \caption{The top part of the balloon, which is a hemisphere}
    \label{fig:1}
\end{figure}

\subsection*{Curved Truncated Cone (Bottom Part)}
The curved truncated cone connects smoothly to the hemisphere at \( z = 0 \) with a top radius of \( R = 10 \) meters and tapers down. The radius at any height \( z \) is given by:
\[
r(z) = R \left(1 - \left(\frac{z}{H}\right)^2\right)
\]

The parametric equations for the truncated cone are:
\[
x = \left(R \left(1 - \left(\frac{z}{H}\right)^2\right)\right) \cos(\theta)
\]
\[
y = \left(R \left(1 - \left(\frac{z}{H}\right)^2\right)\right) \sin(\theta)
\]
\[
z = -z
\]
where \( 0 \leq \theta \leq 2\pi \) and  \( -20 \leq z \leq 0 \)

\begin{figure}[H]
    \centering
    \includegraphics[scale=0.25]{cone.png}
    \caption{The bottom part of the balloon, which is a curved, truncated cone}
    \label{fig:2}
\end{figure}

\subsection*{Combined Shape}
Together, the hemisphere and the curved truncated cone create a smooth balloon shape.

\begin{figure}[H]
    \centering
    \includegraphics[scale=0.4]{balloon.png}
    \caption{Combined hemisphere and truncated cone}
    \label{fig:3}
\end{figure}
\begin{figure}[H]
    \centering
    \includegraphics[scale=0.33]{prompt.png}
    \caption{Specific GeoGebra 3D input to produce both shapes}
    \label{fig:4}
\end{figure}

\section*{Volume Calculations}

\subsection*{Volume of the Hemisphere}
To calculate the volume of the hemisphere with radius \( R \) using spherical coordinates (\( \rho, \theta, \phi \)), we set up the integral for a hemisphere where \( \rho \) ranges from \( 0 \) to \( R \), \( \theta \) ranges from \( 0 \) to \( 2\pi \), and \( \phi \) ranges from \( 0 \) to \( \frac{\pi}{2} \).


\[
V_{\text{hemisphere}} = \int_0^{2\pi} \int_0^{\pi/2} \int_0^R \rho^2 \sin \phi \, d\rho \, d\phi \, d\theta
\]


\[
V_{\text{hemisphere}} = \int_0^{2\pi} \int_0^{\pi/2} \left[ \frac{\rho^3}{3} \right]_0^R \sin \phi \, d\phi \, d\theta
\]

\[
V_{\text{hemisphere}} = \int_0^{2\pi} \int_0^{\pi/2} \frac{R^3}{3} \sin \phi \, d\phi \, d\theta
\]


\[ V_{\text{hemisphere}} =
\frac{R^3}{3} \int_0^{2\pi} d\theta \int_0^{\pi/2} \sin \phi \, d\phi
\]


\[
V_{\text{hemisphere}} = \frac{R^3}{3} \cdot 2\pi \int_0^{\pi/2} \sin \phi \, d\phi
\]


\[
V_{\text{hemisphere}} = \frac{R^3}{3} \cdot 2\pi \left[ -\cos \phi \right]_0^{\pi/2} = \frac{R^3}{3} \cdot 2\pi \left( 0 - (-1) \right) = \frac{2}{3} \pi R^3
\]







Substitute \( R = 10 \):
\[
V_{\text{hemisphere}} = \frac{2}{3} \pi (10)^3 = \frac{2}{3} \cdot 1000 \pi \approx 2094.4 \, \text{m} ^3
\]

\subsection*{Volume of the Curved Truncated Cone}
For the truncated cone with a varying radius, cylindrical coordinates will be used. Let \( R \) be the top radius, \( H \) be the height, and \( r(z) = R \left(1 - \frac{z^2}{H^2}\right) \) be the radius at any height \( z \).

The volume is:
\[
V_{\text{truncated cone}} = \int_0^{2\pi} \int_0^H \int_0^{r(z)} r \, dr \, dz \, d\theta
\]

\[
V_{\text{truncated cone}} = \int_0^{2\pi} \int_0^H  \frac{r^2}{2} \Big|_0^{r(z)} \, dz \, d\theta
\]

\[
V_{\text{truncated cone}} = \int_0^{2\pi} \int_0^H \frac{(r(z))^2}{2} \, dz \, d\theta
\]

\[
V_{\text{truncated cone}} = \int_0^{2\pi} d\theta \int_0^H \frac{R^2}{2} \left(1 - \frac{z^2}{H^2}\right)^2 \, dz
\]

\[
V_{\text{truncated cone}} = \pi R^2 \int_0^H \left(1 - \frac{z^2}{H^2}\right)^2 \, dz
\]


\[
V_{\text{truncated cone}} = \pi R^2 \int_0^H \left(1 - \frac{2z^2}{H^2} + \frac{z^4}{H^4}\right) \, dz
\]

\[
V_{\text{truncated cone}} = \pi R^2 \left[ \int_0^H 1 \, dz - \int_0^H \frac{2z^2}{H^2} \, dz + \int_0^H \frac{z^4}{H^4} \, dz \right]
\]

\[
V_{\text{truncated cone}} = \pi R^2 \left[ H - \left(\frac{2}{H^2}\right)\left(\frac{z^3}{3}\Big|_0^{H}\right) + \left(\frac{1}{H^4}\right) \left(\frac{z^5}{5}\Big|_0^{H}\right)\right]
\]


\[
V_{\text{truncated cone}} = \pi R^2 \left( H - \frac{2H}{3} + \frac{H}{5} \right) = \pi R^2 \cdot \frac{8H}{15}
\]

Substitute \( R = 10 \) and \( H = 20 \):
\[
V_{\text{truncated cone}} = \frac{8}{15} \pi (10)^2 \cdot 20 = \frac{8}{15} \cdot 2000 \pi \approx 3351.0 \, \text{m} ^3
\]

\subsection*{Total Volume of the Balloon}
The total volume of the balloon is the sum of the volumes of the hemisphere and truncated cone:
\[
V_{\text{balloon}} = V_{\text{hemisphere}} + V_{\text{truncated cone}} \approx 2094.4 + 3351.0 = 5445.4  \, \text{cubic meters} 
\]
%https://www.balloon-rides.com/atics-facts.htm#:~:text=A%20typical%20balloon%20is%20about,60%20to%2080%20feet%20tall.


\subsection*{Weight of the Balloon}

As stated earlier, the weight of the cabin is estimated to be approximately \( W_{\text{cabin}} = 45 \, \text{kg} \). According to hot air balloon manufacturing company \textit{Apex Balloons}, urethane-coated nylon would be an excellent choice as a balloon fabric, which has a weight of approximately \( 9 \, \text{oz/yd}^2 \) or \( 0.305 \, \text{kg/m}^2 \). To obtain the weight, surface area needs to be calculated.

The total surface area of the balloon is the combined surface area of the hemisphere and truncated cone that form its shape. By knowing the surface area and the fabric’s weight per unit area, we can calculate the fabric's contribution to the overall weight of the balloon system. 





\section*{Surface Area of the Hemisphere}

For a hemisphere of radius \( R = 10 \), the surface lies above the \( xy \)-plane and can be described by the equation:
\[
x^2 + y^2 + z^2 = R^2
\]

\[
z = \sqrt{R^2 - x^2 - y^2}
\]
Since we are only looking at the upper hemisphere, we use the positive square root.
The surface area \( A \) of a surface \( z = f(x, y) \) over a region \( D \) in the \( xy \)-plane is given by:
\[
A_{\text{hemisphere}} = \iint_D \sqrt{1 + \left( \frac{\partial z}{\partial x} \right)^2 + \left( \frac{\partial z}{\partial y} \right)^2} \, dx \, dy
\]
In our case, \( f(x, y) = \sqrt{R^2 - x^2 - y^2} \).

We first compute the partial derivatives of \( z \) with respect to \( x \) and \( y \):
\[
\frac{\partial z}{\partial x} = \frac{-x}{\sqrt{R^2 - x^2 - y^2}}
\]

\[
\frac{\partial z}{\partial y} = \frac{-y}{\sqrt{R^2 - x^2 - y^2}}
\]

Now, we substitute these partial derivatives:
\[
A_{\text{hemisphere}} = \iint_D \sqrt{1 + \left( \frac{-x}{\sqrt{R^2 - x^2 - y^2}} \right)^2 + \left( \frac{-y}{\sqrt{R^2 - x^2 - y^2}} \right)^2} \, dx \, dy
\]

\[
A_{\text{hemisphere}} = \iint_D \sqrt{1 + \frac{x^2}{R^2 - x^2 - y^2} + \frac{y^2}{R^2 - x^2 - y^2}} \, dx \, dy
\]
\[
A_{\text{hemisphere}} = \iint_D \sqrt{\frac{R^2 - x^2 - y^2}{R^2 - x^2 - y^2} + \frac{x^2}{R^2 - x^2 - y^2} + \frac{y^2}{R^2 - x^2 - y^2}} \, dx \, dy
\]
\[
A_{\text{hemisphere}} = \iint_D \sqrt{\frac{R^2}{R^2 - x^2 - y^2}} \, dx \, dy
\]

\[
A_{\text{hemisphere}} = \iint_D \frac{R}{\sqrt{R^2 - x^2 - y^2}} \, dx \, dy
\]


The region \( D \) is the projection of the hemisphere onto the \( xy \)-plane, which is a disk of radius \( R = 10 \). Thus, we can describe \( D \) in polar coordinates with \( r \) ranging from \( 0 \) to \( R \) and \( \theta \) from \( 0 \) to \( 2\pi \).

\[
x = r \cos \theta, \quad y = r \sin \theta
\]
\[
dx \, dy = r \, dr \, d\theta
\]
Now the integral becomes:
\[
A_{\text{hemisphere}} = \int_0^{2\pi} \int_0^{R} \frac{R}{\sqrt{R^2 - r^2}} \cdot r \, dr \, d\theta
\]

\[
A_{\text{hemisphere}} = R \int_0^{2\pi} d\theta \int_0^{R} \frac{r}{\sqrt{R^2 - r^2}} \, dr
\]

\[
\int_0^{2\pi} d\theta = 2\pi
\]


Let \( u = R^2 - r^2 \), so \( du = -2r \, dr \) or \( r \, dr = -\frac{du}{2} \).
When \( r = 0 \), \( u = R^2 \).
When \( r = R \), \( u = 0 \).

\[
\int_0^{R} \frac{r}{\sqrt{R^2 - r^2}} \, dr = \int_{R^2}^0 \frac{-1}{2 \sqrt{u}} \, du = \frac{1}{2} \int_0^{R^2} u^{-\frac{1}{2}} \, du
\]

\[
\frac{1}{2} \int_0^{R^2} u^{-\frac{1}{2}} \, du = \frac{1}{2} \cdot 2 u^{\frac{1}{2}} \Big|_0^{R^2} = \sqrt{u} \Big|_0^{R^2} = R
\]

Thus, the integral evaluates to \( R \), and the total area becomes:
\[
A_{\text{hemisphere}} = R \cdot 2\pi \cdot R = 2\pi R^2
\]
Note that since the hemisphere is connected directly to the cone, the base area of $\pi r^2$ is not considered. For \( R = 10 \), the surface area of the hemisphere is:
\[
A_{\text{hemisphere}} = 2\pi (10)^2 = 200\pi \approx 628.3 \ \text{m} ^2 
\] 
\subsection*{Surface Area of the Cone}
The surface area of a truncated cone can be calculated using parametric surface area. For a surface parameterized by two variables, the surface area is given by:

\[
A = \iint_D \left| \frac{\partial \mathbf{r}}{\partial u} \times \frac{\partial \mathbf{r}}{\partial v} \right| \, du \, dv
\]

More specifically, if 
\[
f = \langle x, y, z \rangle
\]
then
\[
A = \iint_D \left| \frac{\partial f}{\partial z} \times \frac{\partial f}{\partial \theta} \right| \, dz \, d\theta
\]

where the parametric equations for the surface are:

\[
x = r(z) \cos(\theta), \quad y = r(z) \sin(\theta), \quad z = z
\]

where \( r(z) \) is the radius of the cone at height \( z \), and \( \theta \) is the angle around the cone from \( 0 \) to \( 2\pi \). The radius \( r(z) \) changes with height \( z \), as

\[
r(z) = R \left(1 - \frac{z^2}{H^2}\right) = R - \frac{Rz^2}{H^2}
\]

where \( R \) is the radius at the top of the cone, and \( H \) is the height of the cone.

Now, let's compute the partial derivatives of \( f = \langle x, y, z \rangle \):



\[
\frac{\partial f}{\partial z} = \langle \frac{dr}{dz} \cos(\theta), \frac{dr}{dz} \sin(\theta), 1 \rangle
\]
Taking the derivative of \( r(z) = R - \frac{Rz^2}{H^2} \) with respect to \( z \):

\[
\frac{dr}{dz} = \frac{-2Rz}{H^2}
\]

So,
\[
\frac{\partial f}{\partial z} = \langle \frac{-2Rz}{H^2} \cos(\theta), \frac{-2Rz}{H^2} \sin(\theta), 1 \rangle
\]

As $r$ is not a function of $\theta$, 

\[
\frac{\partial f}{\partial \theta} = \langle -r(z) \sin(\theta), r(z) \cos(\theta), 0 \rangle
\]

Now, we can find

\[
\frac{\partial f}{\partial z} \times \frac{\partial f}{\partial \theta} = \left| \begin{matrix} \mathbf{i} & \mathbf{j} & \mathbf{k} \\ \frac{-2Rz}{H^2} \cos(\theta) & \frac{-2Rz}{H^2} \sin(\theta) & 1 \\ -r(z) \sin(\theta) & r(z) \cos(\theta) & 0 \end{matrix} \right|
\]

Expanding the determinant:

\[
\mathbf{i} \left( \left( \frac{-2Rz}{H^2} \sin(\theta) \cdot 0 - 1 \cdot r(z) \cos(\theta) \right) \right)
\]

\[
- \mathbf{j} \left( \left( \frac{-2Rz}{H^2} \cos(\theta) \cdot 0 - 1 \cdot (-r(z) \sin(\theta)) \right) \right)
\]

\[
+ \mathbf{k} \left( \left( \frac{-2Rz}{H^2} \cos(\theta) \cdot r(z) \cos(\theta) - (\frac{-2Rz}{H^2}) \sin(\theta) \cdot (-r(z) \sin(\theta)) \right) \right)
\]

Simplifying each component:

\[
= \mathbf{i} \left( -r(z) \cos(\theta) \right)
\]

\[
- \mathbf{j} \left( r(z) \sin(\theta) \right)
\]

\[
+ \mathbf{k} \left( r(z) \cdot \frac{-2Rz}{H^2} \left( \cos^2(\theta) + \sin^2(\theta) \right) \right)
\]



\[
= \mathbf{i} \left( -r(z) \cos(\theta) \right)
- \mathbf{j} \left( r(z) \sin(\theta) \right)
+ \mathbf{k} \left( r(z) \cdot \frac{-2Rz}{H^2} \right)
\]

Now, the magnitude can be calculated:

\[
\left| \frac{\partial f}{\partial z} \times \frac{\partial f}{\partial \theta} \right| = \sqrt{ \left( r(z) \cos(\theta) \right)^2 + \left( r(z) \sin(\theta) \right)^2 + \left( r(z) \cdot \frac{2Rz}{H^2} \right)^2 }
\]

\[
\left| \frac{\partial f}{\partial z} \times \frac{\partial f}{\partial \theta} \right| = \sqrt{ r(z)^2 \left( \cos^2(\theta) + \sin^2(\theta) \right) + \left( \frac{2Rz}{H^2} \right)^2 r(z)^2 }
\]

\[
\left| \frac{\partial f}{\partial z} \times \frac{\partial f}{\partial \theta} \right| = \sqrt{ r(z)^2 + \left( \frac{2Rz}{H^2} \right)^2 r(z)^2 }
\]


\[
\left| \frac{\partial f}{\partial z} \times \frac{\partial f}{\partial \theta} \right| = r(z) \sqrt{ 1 + \left( \frac{2Rz}{H^2} \right)^2 }
\]

Thus, the surface area integral becomes:

\[
A_{\text{cone}} = \int_0^{2\pi} \int_0^H r(z) \sqrt{ 1 + \left( \frac{2Rz}{H^2} \right)^2 } \, dz \, d\theta
\]

Substituting \( r(z) = R \left( 1 - \frac{z^2}{H^2} \right) \):

\[
A_{\text{cone}} = \int_0^{2\pi} \int_0^H R \left( 1 - \frac{z^2}{H^2} \right) \sqrt{ 1 + \left( \frac{2Rz}{H^2} \right)^2 } \, dz \, d\theta
\]

\[
A_{\text{cone}} = 2\pi \int_0^H R \left( 1 - \frac{z^2}{H^2} \right) \sqrt{ 1 + \left( \frac{2Rz}{H^2} \right)^2 } \, dz \
\]

\[
A_{\text{cone}} = 2\pi \int_0^H R \left( 1 - \frac{z^2}{H^2} \right) \sqrt{ 1 +  \frac{4R^2z^2}{H^4} } \, dz \
\]


\[
A_{\text{cone}} = 2\pi \int_0^H R \left( 1 - \frac{z^2}{H^2} \right) \sqrt{ \frac{H^4}{H^4} +  \frac{4R^2z^2}{H^4} } \, dz \
\]

\[
A_{\text{cone}} = 2\pi \int_0^H R \left( 1 - \frac{z^2}{H^2} \right) \frac{\sqrt{H^4 + 4R^2z^2}}{H^2}  \, dz \
\]

\[
A_{\text{cone}} = \frac{2\pi R}{H^2} \int_0^H \left( 1 - \frac{z^2}{H^2} \right) \sqrt{H^4 + 4R^2z^2} \ dz \
\]
Here, let $z = \frac{H^2 \tan (x)}{2R}$. Then $dz = \frac{H^2 \sec^2(x)}{2R} dx$, and when $z = H$ then $H = \frac{H^2 \tan(x)}{2R}$ so $0\leq x \leq \tan^{-1}(\frac{2R}{H})$



\[
A_{\text{cone}} = \pi \int_0^{\tan^{-1}(\frac{2R}{H})} \left( 1 - \frac{H^4\tan^2(x)}{4R^2H^2} \right) \sqrt{H^4 + \frac{4R^2H^4\tan^2(x)}{4R^2}} \ \sec^2(x)dx\
\]



\[
A_{\text{cone}} = \pi \int_0^{\tan^{-1}(\frac{2R}{H})} \left( 1 - \frac{H^4\tan^2(x)}{4R^2H^2} \right) \sqrt{H^4 + H^4\tan^2(x)} \ \sec^2(x)dx\
\]

\[
A_{\text{cone}} = \pi H^2 \int_0^{\tan^{-1}(\frac{2R}{H})} \left( 1 - \frac{H^4\tan^2(x)}{4R^2H^2} \right) \sqrt{1 + \tan^2(x)} \ \sec^2(x)dx\
\]


\[
A_{\text{cone}} = \pi H^2 \int_0^{\tan^{-1}(\frac{2R}{H})} \left( 1 - \frac{H^2\tan^2(x)}{4R^2} \right)  \ \sec^3(x)dx\
\]


\[
A_{\text{cone}} = \pi H^2 \left( \int_0^{\tan^{-1}(\frac{2R}{H})} \sec^3(x) dx - \frac{H^2}{4R^2}\int_0^{\tan^{-1}(\frac{2R}{H})} \tan^2(x)\sec^3(x) dx\right ) 
\]
 

\[
A_{\text{cone}} = \pi H^2 \left( \frac{\sec(x) \tan(x) + \ln(|\sec(x) + \tan(x)|)}{2} \Big|_0^{\tan^{-1}\left(\frac{2R}{H}\right)} \right)
\]
\[
- \frac{\pi H^4}{4R^2} \left( \frac{ \tan(x) \sec(x) \left( 2 \sec^2(x) - 1 \right) - \tanh^{-1}(\sin(x))}{8}\Big|_0^{\tan^{-1}\left(\frac{2R}{H}\right)}\right)
\]
When $R = 10$ and $H = 20$, $\tan^{-1}\left(\frac{2R}{H}\right) = \frac{\pi}{4}$, so this evaluates to 
\[400\pi\left(\frac{\sqrt2 + \ln(1 + \sqrt2)}{2}\right) - 6.25\pi \left( \frac{3\sqrt2 - \coth^{-1}(\sqrt2)}{8} \right) \approx 1434.12 \ \text{m} ^2 \] according to Wolfram Alpha.

Thus, the total weight is the surface area multiplied by the weight of the material per square meter, then added with the weight of the cabin, which is $0.305(628.3
+ 1434.12) + 45 = 674.0381$ kg. Accounting for the weight of the passengers means that the total weight would be $674.0381 + 362.9 = 1036.9381 \ \text{kg}$ 


\subsection*{Temperature of Hot Air as a Function of Altitude}
The temperature of the hot air inside a hot air balloon as a function of altitude can be derived from the principles of thermodynamics and atmospheric behavior. As a hot air balloon ascends, the pressure decreases and the temperature of the air inside the balloon decreases as well, due to the atmosphere's lapse rate. However, since the air inside the balloon is heated to provide lift, it will not follow the typical atmospheric temperature gradient directly but will still be affected by altitude.
%https://www.britannica.com/science/lapse-rate
The ideal gas law governs the relationship between pressure, temperature, and volume in the balloon. As the balloon ascends, both the pressure and temperature of the hot air inside the balloon will change, but the air is kept hot by heating systems. The temperature of the surrounding air decreases with altitude according to the atmospheric lapse rate.

The atmospheric lapse rate \( \lambda \) is typically about \( 0.00649 \, \text{K/m} \), which means that for every meter of altitude gained, the temperature decreases by approximately \( 0.00649 \, \text{K} \). The temperature of the air inside the balloon must remain higher than the surrounding air to maintain buoyancy. The general formula for the temperature of the hot air in the balloon as a function of altitude \( h \) is:

\[
T_h(h) = T_0 - \lambda h
\]

where \( T_h(h) \) is the temperature of the hot air at altitude \( h \), \( T_0 \) is the temperature of the hot air at sea level,
\( \lambda = 0.00649 \, \text{K/m} \) is the atmospheric lapse rate, and \( h \) is the altitude in meters.

In preparation for the analysis of how different scale factors of volume influence temperature, assume that the relationship between temperature and balloon size can be modeled using a linear adjustment formula. In reality, the relationship is based on complex thermodynamics, but for simplicity we use a linear model.

The Ideal Gas Law states that:

\[
PV = nRT
\]

where \( P \) is the pressure, \( V \) is the volume, \( n \) is the number of moles of gas, \( R \) is the ideal gas constant, and \( T \) is the temperature in Kelvin.

For a balloon that is open to the atmosphere, the pressure inside and outside the balloon is approximately equal, so we can assume that the pressure remains constant. Therefore, from the Ideal Gas Law, we can derive that:

\[
T = \frac{PV}{nR}
\]

Since \( P \), \( n \), and \( R \) are constant, the temperature inside the balloon is directly proportional to the volume of the balloon. This means, for a fixed amount of air, as the volume increases, the temperature must increase to maintain buoyancy.

This relationship can be simplified using Charles’s Law, which states:

\[
\frac{V_1}{T_1} = \frac{V_2}{T_2}
\]

where \( V_1 \) and \( T_1 \) are the initial volume and temperature, and \( V_2 \) and \( T_2 \) are the final volume and temperature. This equation tells us that for a fixed pressure, volume and temperature are directly proportional. Therefore, if the volume of the balloon increases, the temperature must increase in proportion to that increase in volume.



The temperature inside the balloon at a given volume can be modeled using the following equation:

\[
T_h(k) = T_0 \cdot k
\]

where \( T_h(k) \) is the temperature inside the balloon for a given size factor \( k \), \( T_0 \) is the temperature at the base size factor \( k = 1 \), and \( k \) is the size factor of the balloon.



This adjustment is reasonable because it follows from both the Ideal Gas Law and Charles’s Law, both of which describe the relationship between pressure, temperature, and volume. The volume of the balloon is directly related to the amount of air inside, and since the temperature of the air must increase to keep the air less dense than the surrounding air, the formula appropriately models this relationship.





\subsection*{Selecting Altitude}
When choosing an appropriate cruising altitude for the hot air balloon, several factors need to be considered, including the weight and volume of the balloon, the temperature of the hot air required for lift, the safety of the material, and the terrain over which the balloon will be traveling. In this case, the balloon weighs approximately 1037 kg with passengers, has a volume of 5445 cubic meters, and is made of urethane-coated nylon, an important factor for determining optimal temperatures. The balloon will be flying over Irvine, California, at an average elevation of 56 feet (17 meters) above sea level.

The altitude for the balloon ride should balance lift generation and temperature control. As the altitude increases, the atmospheric pressure decreases, causing the air inside the balloon to cool and expand. This means that a higher altitude requires a higher temperature inside the balloon to maintain lift, but the fabric material of the balloon (urethane-coated nylon) can only withstand temperatures below 493.15 K before it starts melting.

Open source data provided by \textit{The Engineering ToolBox} indicates that the temperature of the air at sea level is approximately 288 K, with the temperature decreasing at an average rate of 0.00649 K per meter of altitude. These constants can be used to calculate the temperature at various altitudes. For example, at an altitude of 1000 meters, the temperature inside the balloon would be approximately:

\[
T_h(1000) = 288 \, \text{K} - 0.00649 \times 1000 = 284.51 \, \text{K}
\]

This temperature is well within the safe operating range for the nylon fabric. Therefore, cruising at 1000 meters is feasible and ideal for ensuring passenger comfort and maintaining buoyancy.

Considering the local weather and terrain, a cruising altitude of 1000 meters would be optimal for this balloon. This altitude provides a comfortable environment for the passengers while ensuring that the balloon’s lift is sufficiently stable. 

While increasing altitude is an important factor in hot air ballooning, the weather conditions can significantly impact the safety and comfort of the ride. Hot air ballooning is heavily influenced by wind patterns, air density, and thermal updrafts, all of which can affect the stability of the balloon and the ability to maintain a constant altitude. At higher altitudes, the wind can be stronger, making it more challenging to control the balloon’s trajectory. To mitigate this, 1000 meters is selected, not too low as to not provide any entertainment value, but not too high as to face unpredictable wind currents.

\subsection*{Different Size Factors $k$ with Corresponding Temperatures}
Here are different size factors of the balloon and their corresponding temperature.
\begin{table}[ht]
\centering
\begin{tabular}{|c|c|c|}
\hline
\textbf{Size Factor \( k \)} & \textbf{Volume \( V \) (\( \text{m}^3 \))} & \textbf{Temperature \( T_h \) (K) at \( h = 1000 \, \text{m} \)} \\
\hline
1 & 5445.4 & 284.5 \\
1.5 & 8168.1 & 426.75 \\
2 & 10890.8 & 569 \\
2.5 & 13613.5 & 711.25 \\
3 & 16336.2 & 853.5 \\
3.5 & 19058.9 & 997.75 \\
4 & 21781.6 & 1138 \\
4.5 & 24404.3 & 1280.25 \\
5 & 27227.0 & 1422.5 \\
\hline
\end{tabular}
\caption{Balloon Size and Corresponding Hot Air Temperature at 1000 meters}
\end{table}

Evidently, temperature steadily increases as the size factor $k$ increases in a linear fashion. Recall that the melting point for the balloon fabric is 493.15 K. Based on this, a size factor of 1.5 would likely be the best option, maximizing size while minimizing risk of fabric overheat.

\subsection*{Conclusion}
In conclusion, a balloon has been successfully modeled by combining a hemisphere and a curved cone. Based on volume calculations, an optimal cruising altitude and size factor were determined. While all analysis was done in a theoretical fashion, in reality there are many complex thermodynamic relationships that this analysis does not consider. Additionally, having a 2 ton hot air balloon, while not exactly impossible, could be unfeasible to implement. 


\subsection*{Bibliography}

eRCadmin. “How Big Is a Hot Air Balloon Basket? (Answered).” Napa Valley Aloft, 18 Sept. 2023, nvaloft.com/2023/09/18/how-big-is-a-hot-air-balloon-basket/.
\\
\\
“A-Type – Cameron Balloons US.” Cameronballoons.com, 2016, cameronballoons.com/products/ \\
hot-air-balloons/envelopes/a-type. 
\\
\\
“APEX Balloons - Hot Air Balloon Manufacturer, Hot Air Airships, Balloon Repair Station.” Www.apexballoons.com, www.apexballoons.com/balloons/.
\\
\\
Engineering ToolBox. “U.S. Standard Atmosphere.” Engineeringtoolbox.com, 2003, $www.engineeringtoolbox.com/standard-atmosphere-d_604.html.$
\\
\\
Wikipedia Contributors. “Nylon.” Wikipedia, Wikimedia Foundation, 24 Oct. 2019, en.wikipedia.org/wiki/Nylon.


\end{document}



